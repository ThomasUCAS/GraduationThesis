\chapter{引言}\label{chap:introduction}

\section{研究背景}

考虑到许多同学可能缺乏\LaTeX{}使用经验,ucasthesis将\LaTeX{}的复杂性高度封装,开放出简单的接口,以便轻易使用。同时,对用\LaTeX{}撰写论文的一些主要难题,如制图、制表、文献索引等,进行了详细说明,并提供了相应的代码样本,理解了上述问题后,对于初学者而言,使用此模板撰写学位论文将不存在实质性的困难。所以,如果你是初学者,请不要直接放弃,因为同样为初学者的我,十分明白让\LaTeX{}简单易用的重要性,而这正是ucasthesis所追求和体现的。

此中国科学院大学学位论文模板ucasthesis基于中科院数学与系统科学研究院吴凌云研究员的CASthesis模板发展而来。当前ucasthesis模板满足最新的中国科学院大学学位论文撰写要求和封面设定。兼顾操作系统:Windows,Linux,MacOS 和\LaTeX{}编译引擎:pdflatex,xelatex,lualatex。支持中文书签、中文渲染、中文粗体显示、拷贝PDF中的文本到其他文本编辑器等特性。此外,对模板的文档结构进行了精心设计,撰写了编译脚本提高模板的易用性和使用效率。

ucasthesis的目标在于简化学位论文的撰写,利用\LaTeX{}格式与内容分离的特征,模板将格式设计好后,作者可只需关注论文内容。 同时,ucasthesis有着整洁一致的代码结构和扼要的注解,对文档的仔细阅读可为初学者提供一个学习\LaTeX{}的窗口。此外,模板的架构十分注重通用性,事实上,ucasthesis不仅是国科大学位论文模板,同时,通过少量修改即可成为使用\LaTeX{}撰写中英文文章或书籍的通用模板,并为使用者的个性化设定提供了接口。

\section{系统要求}\label{sec:system}

\href{https://github.com/mohuangrui/ucasthesis}{\texttt{ucasthesis}} 宏包可以在目前主流的 \href{https://en.wikibooks.org/wiki/LaTeX/Introduction}{\LaTeX{}} 编译系统中使用,如\TeX{}Live和MiK\TeX{}。因C\TeX{}套装已停止维护,\textbf{不再建议使用} (请勿混淆C\TeX{}套装与ctex宏包。C\TeX{}套装是集成了许多\LaTeX{}组件的\LaTeX{}编译系统。 \href{https://ctan.org/pkg/ctex?lang=en}{ctex} 宏包如同ucasthesis,是\LaTeX{}命令集,其维护状态活跃,并被主流的\LaTeX{}编译系统默认集成,是几乎所有\LaTeX{}中文文档的核心架构)。推荐的 \href{https://en.wikibooks.org/wiki/LaTeX/Installation}{\LaTeX{}编译系统} 和 \href{https://en.wikibooks.org/wiki/LaTeX/Installation}{\LaTeX{}文本编辑器} 为


\LaTeX{}编译系统,如\TeX{}Live(Mac\TeX{}为针对MacOS的\TeX{}Live),用于提供编译环境,\LaTeX{}文本编辑器 (如Texmaker) 用于编辑\TeX{}源文件。请从各软件官网下载安装程序,勿使用不明程序源。\textbf{\LaTeX{}编译系统和\LaTeX{}编辑器分别安装成功后,即完成了\LaTeX{}的系统配置},无需其他手动干预和配置。若系统原带有旧版的\LaTeX{}编译系统并想安装新版,请\textbf{先卸载干净旧版再安装新版}。

\section{问题反馈}

请见 \href{https://github.com/mohuangrui/ucasthesis/wiki/%E5%B8%B8%E8%A7%81%E9%97%AE%E9%A2%98}{问题反馈} 

欢迎大家有效地反馈模板不足之处,一起不断改进模板。希望大家向同事积极推广\LaTeX{},一起更高效地做科研。
